% !TEX TS-program = pdflatex
% !TEX encoding = UTF-8 Unicode

% This is a simple template for a LaTeX document using the "article" class.
% See "book", "report", "letter" for other types of document.

\documentclass[12pt]{article} % use larger type; default would be 10pt

\usepackage[utf8]{inputenc} % set input encoding (not needed with XeLaTeX)

%%% Examples of Article customizations
% These packages are optional, depending whether you want the features they provide.
% See the LaTeX Companion or other references for full information.

%%% PAGE DIMENSIONS
\usepackage{geometry} % to change the page dimensions
\geometry{a4paper} % or letterpaper (US) or a5paper or....
\geometry{margin=1in} % for example, change the margins to 2 inches all round
% \geometry{landscape} % set up the page for landscape
%   read geometry.pdf for detailed page layout information

\usepackage{graphicx} % support the \includegraphics command and options

%%% PACKAGES
\usepackage{booktabs} % for much better looking tables
\usepackage{array} % for better arrays (eg matrices) in maths
\usepackage{paralist} % very flexible & customisable lists (eg. enumerate/itemize, etc.)
\usepackage{verbatim} % adds environment for commenting out blocks of text & for better verbatim
%\usepackage{subfig} % make it possible to include more than one captioned figure/table in a single float
\usepackage{pdfpages}
\usepackage{amsmath,mathtools,amsfonts}
\usepackage{bm}
\usepackage{blindtext}
\usepackage{fancyhdr}
\usepackage{float}
\usepackage{gensymb}
\usepackage{svg}
\usepackage{subcaption}
\usepackage{listings}
\usepackage{color}

\definecolor{mygreen}{rgb}{0,0.6,0}
\definecolor{mygray}{rgb}{0.5,0.5,0.5}
\definecolor{mymauve}{rgb}{0.58,0,0.82}

\lstset{ 
	backgroundcolor=\color{white},   % choose the background color; you must add \usepackage{color} or \usepackage{xcolor}; should come as last argument
	basicstyle=\footnotesize,        % the size of the fonts that are used for the code
	breakatwhitespace=false,         % sets if automatic breaks should only happen at whitespace
	breaklines=true,                 % sets automatic line breaking
	captionpos=b,                    % sets the caption-position to bottom
	commentstyle=\color{mygreen},    % comment style
	deletekeywords={...},            % if you want to delete keywords from the given language
	escapeinside={\%*}{*)},          % if you want to add LaTeX within your code
	extendedchars=true,              % lets you use non-ASCII characters; for 8-bits encodings only, does not work with UTF-8
	firstnumber=001,                % start line enumeration with line 1000
	frame=single,	                   % adds a frame around the code
	keepspaces=true,                 % keeps spaces in text, useful for keeping indentation of code (possibly needs columns=flexible)
	keywordstyle=\color{blue},       % keyword style
	language=Octave,                 % the language of the code
	morekeywords={*,...},            % if you want to add more keywords to the set
	numbers=left,                    % where to put the line-numbers; possible values are (none, left, right)
	numbersep=5pt,                   % how far the line-numbers are from the code
	numberstyle=\tiny\color{mygray}, % the style that is used for the line-numbers
	rulecolor=\color{black},         % if not set, the frame-color may be changed on line-breaks within not-black text (e.g. comments (green here))
	showspaces=false,                % show spaces everywhere adding particular underscores; it overrides 'showstringspaces'
	showstringspaces=false,          % underline spaces within strings only
	showtabs=false,                  % show tabs within strings adding particular underscores
	stepnumber=2,                    % the step between two line-numbers. If it's 1, each line will be numbered
	stringstyle=\color{mymauve},     % string literal style
	tabsize=2,	                   % sets default tabsize to 2 spaces
	title=\lstname                   % show the filename of files included with \lstinputlisting; also try caption instead of title
}

\graphicspath{ {./images/} }

%%% HEADERS & FOOTERS
\pagestyle{fancy}
\renewcommand{\headrulewidth}{1pt} % customise the layout...
\renewcommand{\footrulewidth}{1pt} % customise the layout...
\fancyhf{}

% HEADER
\fancyhead[L]{COSC-364}
\fancyhead[C]{}
\fancyhead[R]{Assignment 2}

% FOOTER
\fancyfoot[L]{}
\fancyfoot[C]{\thepage}
\fancyfoot[R]{}

%% TITLE
\title{\uppercase{
	COSC-364 Flow Planning Assignment
}}
\date{\today}
\author{
	Will Cowper\\
	{\small{ID: 81163265}}\\
	{\small{wgc22@uclive.ac.nz}}\\
	{\small{Contribution: 50\%}}\\
	\\
	Jesse Sheehan\\
	{\small{ID: 53366509}}\\
	{\small{jps111@uclive.ac.nz}}\\
	{\small{Contribution: 50\%}}\\
	\\
}

%\renewcommand{\sectionmark}[1]{\markright{\thesection\ #1}}

\begin{document}

\maketitle

\newpage

\section{Problem Formulation}

% The network is defined as having $X$ source nodes ($S_1,\ S_2,\ \ldots,\ S_X$), $Y$ transit nodes ($T_1,\ T_2,\ \ldots,\ T_Y$) and $Z$ destination nodes ($D_1,\ D_2,\ \ldots,\ D_Z$). All transit nodes are directly connected to all source and destination nodes. Source and destination nodes are only connected directly to transit nodes. In this way, all source nodes are connected to all destination nodes via all transit nodes.
% The demand flow between source $i$ and destination $j$ is $h_{ij} = 2i + j$. Additionally, the load from source $i$ to destination $j$ must go through exactly two transit nodes. \\

% As this is a load balancing problem, we are looking to minimize the object function $r$. Where $r$ is the... (IS THIS CORRECT? EXPLAIN) \\

\noindent \textbf{Notation:}
\begin{itemize}
\item $X$ is the number of source nodes.
\item $Y$ is the number of transit nodes.
\item $Z$ is the number of destination nodes.
\item $S_i$ is the $i$th source node.
\item $T_k$ is the $k$th transit node.
\item $D_j$ is the $j$th destination node.
\item $h_{ij}$ is the demand flow between $S_i$ and $D_j$. This is equal to $2i + j$.
\item $c_{ik}$ is the link capacity between $S_i$ and $T_k$.
\item $d_{kj}$ is the link capacity between $T_k$ and $D_j$.
\item $x_{ikj}$ is the decision variable associated with the...
\item $u_{ikj}$ is the binary decision variable associated with the... These are required because $h_{ij}$ must be split across exactly two transit nodes.
\item $l_{k}$ is the load on $T_k$.
\end{itemize}

\subsection{Objective Function}

\begin{align}
\text{minimize}_{[r]}
\end{align}

\subsection{Demand Constraints}

\begin{align}
\sum_{k = 1}^{Y} x_{ikj} &= 2i + j & i \in \{1, \ldots, X\}, j \in \{1, \ldots, Z\}
\end{align}

\subsection{Capacity Constraints}

\begin{align}
\sum_{j = 1}^{Z} x_{ikj} &= c_{ik} & i \in \{1, \ldots, X\}, k \in \{1, \ldots, Y\} \\[1em]
\sum_{i = 1}^{X} x_{ikj} &= d_{kj} & k \in \{1, \ldots, Y\}, j \in \{1, \ldots, Z\} \\[1em]
\sum_{k = 1}^{Y} x_{ikj} &= l_k & i \in \{1, \ldots, X\}, j \in \{1, \ldots, Z\} \\[1em]
\sum_{k = 1}^{Y} u_{ikj} &= 2 & i \in \{1, \ldots, X\}, j \in \{1, \ldots, Z\}
% \sum_{k = 1}^{Y} x_{ikj} &\leq r & i \in \{1, \ldots, X\}, j \in \{1, \ldots, Z\} \\[1em]
\end{align}

\subsection{Non-Negativity Constraints}

\begin{align}
r &\geq 0 \\[1em]
x_{ijk} &\geq 0 & i \in \{1, \ldots, X\}, k \in \{1, \ldots, Y\}, j \in \{1, \ldots, Z\}
\end{align}

% Discuss the different types of constraints and how we define them mathematically

% Demand Constraints

\section{Results}

% Show results for the CPLEX execution time, the number of links with non-zero capacities, the spread of the transit node loads (the difference between the largest and the smallest transit node load), and the highest capacity links, for all varying Y.
% Show these results as a graph or a table.
% Please explain the results

\section{Appendix}


\subsection{Source Code}

\subsubsection{src/\_\_main\_\_.py}
\lstinputlisting[language=Python]{../src/__main__.py}

\subsubsection{src/lp\_utils.py}
\lstinputlisting[language=Python]{../src/lp_utils.py}

\subsubsection{src/lp\_gen.py}
\lstinputlisting[language=Python]{../src/lp_gen.py}

% Add any other source files we've used

\subsection{Generated LP File}

\subsubsection{problem\_3\_2\_4.lp}
% \lstinputlisting[language=Python]{../src/__main__.py}

% Ask Andreas about this one! The parameters of this file directly contradict the spec!

\subsection{Plagiarism Declaration}

% insert the signed plagiarism declaration (possible as an include pdf command just after the title page

\end{document}
